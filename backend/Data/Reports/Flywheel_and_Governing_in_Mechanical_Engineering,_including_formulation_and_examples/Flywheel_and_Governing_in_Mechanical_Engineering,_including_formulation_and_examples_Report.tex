\documentclass[11pt,a4paper]{article}
\usepackage[top=2.5cm, bottom=2.5cm, left=3cm, right=3cm]{geometry}
\usepackage{graphicx}
\usepackage{amsmath, amssymb}
\usepackage{tabularx}
\usepackage{booktabs}
\usepackage{multirow}
\usepackage{ragged2e}
\usepackage{array}
\usepackage{setspace}
\usepackage{float}
\usepackage{microtype}
\onehalfspacing
\begin{document}

\begin{titlepage}
    \centering
    \vspace*{1cm}
    {\Huge\bfseries Comprehensive Report on Flywheel and Governing in Mechanical Engineering\par}
    \vspace{1.5cm}
    {\Large\itshape Subhash Chandra\par}
    \vspace{0.5cm}
    {\large MTech (Thermal Engineering)\\
    Govind Ballabh Pant Institute of Engineering \& Technology (GBPIET)\\
    College ID: 245903\par}
    \vspace{1cm}
    {\large \today\par}
    \vfill
    \begin{abstract}
        \noindent This report provides a comprehensive analysis of the flywheel as a fundamental mechanical device for energy storage and speed regulation, or governing, in dynamic systems. The core principle explored is the flywheel's function as an inertial reservoir that absorbs and releases kinetic energy to mitigate fluctuations in torque and rotational speed, thereby ensuring smoother operation of machinery. The document systematically derives the governing mathematical formulations, starting from the basic kinetic energy equation to the detailed design equations involving the coefficient of fluctuation of speed and energy. A detailed comparative analysis of materials and system types is presented. Furthermore, the report connects classical theory with modern, high-technology applications such as grid-scale energy storage and hybrid systems, illustrating the evolving relevance of flywheel technology. Practical design procedures and numerical examples are included to bridge theory with application, making this a complete reference for understanding flywheel design and governing in mechanical engineering.
    \end{abstract}
    \vfill
\end{titlepage}

\tableofcontents
\newpage

\section{Introduction}
In mechanical systems where the driving torque is intermittent or pulsating—such as in internal combustion engines, punching presses, or piston pumps—the speed of the output shaft is prone to undesirable fluctuations. Conversely, in systems powered by renewable but intermittent sources like wind or solar, the energy supply itself is variable. The flywheel addresses both challenges through a singular, elegant principle: the storage of energy in the form of rotational kinetic energy.

A \textbf{flywheel} is a mechanical battery, a rotating mass designed to possess a significant moment of inertia. Its primary function is to act as an \textbf{energy buffer}. During periods when the supplied torque exceeds the load requirement (e.g., the power stroke of an engine), the flywheel absorbs the surplus energy, causing a slight increase in its rotational speed. During periods of torque deficit (e.g., the compression stroke), it releases this stored energy, causing a slight decrease in speed. This cyclic exchange smoothens the torque output and, critically, limits the speed variation of the system.

\textbf{Governing}, in a broader sense, refers to the maintenance of a desired mean operational speed. While an active governor (like a centrifugal governor) regulates fuel or energy input to control average speed, the flywheel performs a complementary \textbf{passive governing} function. It does not maintain a constant speed but rather \textit{minimizes the cyclic fluctuation about the mean speed}. The effectiveness of this function is quantified by the \textbf{Coefficient of Fluctuation of Speed (C$_\text{s}$)}.

This report will establish the theoretical foundation of flywheel operation, derive its essential design equations, detail the design procedure with examples, and explore its advanced applications in modern engineering, particularly relevant to the field of energy systems and thermal engineering.

\section{Theoretical Foundations and Key Concepts}

\subsection{Core Principle: Kinetic Energy Storage}
The fundamental quantity stored in a rotating flywheel is kinetic energy. For a rigid body rotating about a fixed axis, this energy ($E_k$) is given by:
\begin{equation}
    E_k = \frac{1}{2} I \omega^2
\end{equation}
where $I$ is the mass moment of inertia (kg$\cdot$m$^2$) and $\omega$ is the angular velocity (rad/s). This equation reveals the two levers for energy storage: inertia ($I$) and speed ($\omega$). Since energy is proportional to the square of the angular speed, increasing speed is far more effective for boosting energy capacity than increasing mass.

\subsection{Mass Moment of Inertia (I)}
The moment of inertia quantifies a body's resistance to changes in its rotational motion and dictates its energy storage capacity for a given speed. It depends on the mass distribution relative to the axis of rotation. For a flywheel, the geometry is typically designed to concentrate mass at the largest possible radius to maximize $I$. Common formulas include:
\begin{itemize}
    \item \textbf{Solid Disk:} $I = (1/2) m r^2$ where $m$ is mass (kg) and $r$ is outer radius (m).
    \item \textbf{Thin Rim (Approximation):} $I \approx m r^2$, where mass is assumed to be concentrated at the mean rim radius.
    \item \textbf{General Form:} $I = k m r^2$, where $k$ is an inertial constant dependent on geometry (e.g., $k=0.606$ for a flat solid disk).
\end{itemize}

\subsection{The Turning Moment Diagram and Fluctuation of Energy ($\Delta E$)}
The analysis of a flywheel begins with a \textbf{Turning Moment (or Crank Effort) Diagram}, which plots the net torque acting on the crankshaft against the crank angle for one complete cycle.
\begin{itemize}
    \item The \textbf{mean torque ($T_\text{mean}$)} line is drawn such that the area under the torque curve equals the area under the mean line over one cycle (representing constant work output).
    \item The areas between the actual torque curve and the mean torque line represent energy imbalances. Areas above the line indicate surplus energy that accelerates the flywheel (increasing $\omega$), while areas below indicate energy deficit that decelerates it (decreasing $\omega$).
    \item The algebraic sum of these areas through a cycle yields the \textbf{Maximum Fluctuation of Energy ($\Delta E$)}, which is the difference between the maximum and minimum kinetic energies in the flywheel during the cycle. This $\Delta E$ is the critical quantity of energy the flywheel must absorb and release.
\end{itemize}

\subsection{Coefficients of Fluctuation}
Two dimensionless coefficients are vital for design specification:
\begin{enumerate}
    \item \textbf{Coefficient of Fluctuation of Speed ($C_s$):} Defines the permissible speed variation.
    \begin{equation}
        C_s = \frac{\omega_{\text{max}} - \omega_{\text{min}}}{\omega} = \frac{N_{\text{max}} - N_{\text{min}}}{N}
    \end{equation}
    where $\omega$ and $N$ are the mean angular and rotational speeds, respectively. $C_s$ is a design input based on the application's required smoothness (e.g., low for generators, higher for crushers).
    \item \textbf{Coefficient of Fluctuation of Energy ($C_e$):} A property of the prime mover, defined as the ratio of the maximum energy fluctuation to the work done per cycle. It is often determined experimentally from the turning moment diagram.
\end{enumerate}

\subsection{Governing vs. Flywheel Action}
It is crucial to distinguish between a \textbf{Governor} and a \textbf{Flywheel}:
\begin{itemize}
    \item \textbf{Governor:} An \textbf{active feedback control device}. It senses the current speed, compares it to a desired setpoint, and automatically adjusts the energy input (e.g., fuel supply to an engine, steam flow to a turbine) to maintain a constant \textit{mean speed} over varying loads.
    \item \textbf{Flywheel:} A \textbf{passive energy storage device}. It has no sensing or control capability. It automatically smooths out \textit{cyclic speed variations within one or two revolutions} by exchanging kinetic energy with the system, but does not control the long-term average speed.
\end{itemize}
In practice, many systems (like internal combustion engines) employ both: the governor controls the fuel throttle to maintain average speed over minutes, while the flywheel smooths the pulsations from individual power strokes occurring in milliseconds.

\section{Mathematical Formulation and Design Equations}

\subsection{Fundamental Flywheel Sizing Equation}
The core design equation links the energy fluctuation to the flywheel's inertia and the permitted speed fluctuation. Starting from the definition of kinetic energy fluctuation:
\begin{equation}
    \Delta E = E_{\text{max}} - E_{\text{min}} = \frac{1}{2} I (\omega_{\text{max}}^2 - \omega_{\text{min}}^2)
\end{equation}
Factoring the difference of squares and substituting the mean speed $\omega = (\omega_\text{max} + \omega_\text{min})/2$ and the coefficient $C_s$ yields the primary design formula:
\begin{equation}
    \Delta E = I \omega^2 C_s
\end{equation}
Alternatively, since the mean kinetic energy is $E = \frac{1}{2} I \omega^2$, it can also be expressed as:
\begin{equation}
    \Delta E = 2 E C_s
\end{equation}
Therefore, the required mass moment of inertia for a given $\Delta E$ and chosen $C_s$ is:
\begin{equation}
    I = \frac{\Delta E}{\omega^2 C_s}
\end{equation}
This is the pivotal result: \textbf{The flywheel's inertia $I$ is directly proportional to the energy fluctuation $\Delta E$ it must handle and inversely proportional to the square of its operating speed $\omega$ and the allowable speed coefficient $C_s$.}

\subsection{Determination of Maximum Fluctuation of Energy ($\Delta E$)}
For a given turning moment diagram, $\Delta E$ is found graphically or numerically. The process involves:
\begin{enumerate}
    \item Calculating the work done per cycle (area under torque-angle curve).
    \item Finding the mean torque $T_\text{mean} = \text{(Work per cycle)} / (2\pi)$ for a $2\pi$ radian cycle.
    \item Drawing the $T_\text{mean}$ line on the diagram.
    \item Computing the cumulative energy variation at each point relative to the $T_\text{mean}$ line: $E(\theta) = \int (T(\theta) - T_\text{mean}) d\theta$.
    \item Identifying the maximum ($E_\text{max}$) and minimum ($E_\text{min}$) values of this cumulative energy function. Their difference is $\Delta E$.
\end{enumerate}
For multi-cylinder engines, the turning moment diagram is smoother, and $\Delta E$ is smaller, requiring a less massive flywheel.

\subsection{Stress Analysis: The Limiting Factor}
As rotational speed increases, centrifugal forces generate significant tensile stresses within the flywheel material. For a \textit{thin rotating ring} (a common flywheel approximation), the maximum tangential (hoop) stress $\sigma_t$ is:
\begin{equation}
    \sigma_t = \rho \omega^2 r^2 = \rho v^2
\end{equation}
where $\rho$ is the material density (kg/m$^3$), $r$ is the radius (m), and $v$ is the tangential velocity at the rim (m/s). This shows that \textbf{for a given material (with a maximum allowable stress $\sigma_\text{max}$), the maximum safe rim velocity is the critical design limit}. This is why advanced flywheels use high-strength, low-density composites to achieve very high rotational speeds and energy densities.

For a solid disk of constant thickness, the stress distribution is more complex, with maximum stress occurring at the center.

\subsection{Energy and Power in Speed Changes}
The time $t$ required for a flywheel to change speed from $\omega_1$ to $\omega_2$ under the influence of a constant accelerating or braking torque $T$ is given by:
\begin{equation}
    t = \frac{I (\omega_2 - \omega_1)}{T}
\end{equation}
The corresponding energy transferred is:
\begin{equation}
    \Delta E = \frac{1}{2} I (\omega_2^2 - \omega_1^2)
\end{equation}
These equations are essential for analyzing flywheel response in applications like braking energy recovery.

\begin{table}[H]
    \centering
    \caption{Comparison of Flywheel System Types and Material Properties}
    \label{tab:comparison}
    \begin{tabularx}{\textwidth}{|>{\RaggedRight}X|>{\RaggedRight}X|>{\RaggedRight}X|}
        \hline
        \textbf{Feature / Material} & \textbf{Traditional Low-Speed Flywheel (Steel)} & \textbf{Advanced High-Speed Flywheel (Composite)} \\
        \hline
        Typical Rotor Material & Cast iron, Forged steel & Carbon-fiber / Epoxy composite, S-glass composite \\
        \hline
        Typical Geometry & Solid disk or thick rim with spokes & Thin, rim-shaped rotor \\
        \hline
        Operational Speed & Hundreds to a few thousand RPM & 20,000 – 100,000+ RPM \\
        \hline
        Bearing Type & Mechanical rolling-element bearings & Magnetic bearings (active or passive) or superconducting bearings \\
        \hline
        Operating Environment & Air & High vacuum enclosure (to reduce drag) \\
        \hline
        Primary Advantage & Low cost, simple construction, high reliability & Very high energy density, long life, low maintenance \\
        \hline
        Primary Limitation & Low energy-to-mass ratio, high weight, bearing wear & High cost, complex containment for safety, bearing system complexity \\
        \hline
        Common Applications & Engine smoothing, punch presses, basic machinery & Grid frequency regulation, UPS, aerospace, regenerative braking \\
        \hline
    \end{tabularx}
\end{table}

\subsection{Comparison of Flywheel Technologies and Materials}
Table \ref{tab:comparison} synthesizes key characteristics of traditional and modern flywheel systems, highlighting the evolution driven by material science and bearing technology.

\section{Flywheel Design Procedure: A Step-by-Step Methodology}
The systematic design of a flywheel for speed smoothing involves the following steps:
\begin{enumerate}
    \item \textbf{Determine the Turning Moment Diagram:} Obtain or calculate the torque vs. crank-angle relationship for the prime mover (e.g., from engine indicator diagrams or dynamic simulation).
    \item \textbf{Calculate Work per Cycle and Mean Torque:} Integrate the torque over one full cycle to find total work. Mean torque $T_\text{mean} = \text{Work} / (2\pi)$ for a two-revolution (4-stroke) cycle.
    \item \textbf{Determine the Fluctuation of Energy ($\Delta E$):} Using the graphical or numerical method described in Section 3.2, find the maximum and minimum cumulative energy values and compute $\Delta E$.
    \item \textbf{Select the Coefficient of Fluctuation of Speed ($C_s$):} Choose an appropriate $C_s$ based on the application's requirements (e.g., 0.02 for steam engines, 0.2 for crushing machinery).
    \item \textbf{Determine the Mean Operating Speed ($\omega$):} This is typically the nominal speed of the machine (e.g., engine RPM converted to rad/s).
    \item \textbf{Calculate the Required Moment of Inertia ($I$):} Apply the fundamental equation $I = \Delta E / (\omega^2 C_s)$.
    \item \textbf{Define Flywheel Geometry and Dimensions:} Decide on a geometry (solid disk, rim with spokes, etc.). For a rim-type flywheel, where most mass $m$ is in the rim, $I \approx m r_\text{mean}^2$. Choose a suitable mean radius $r_\text{mean}$ based on spatial constraints and calculate the required mass: $m = I / r_\text{mean}^2$.
    \item \textbf{Perform Stress and Safety Analysis:} Calculate the maximum tangential stress (using $\sigma_t = \rho \omega^2 r^2$ for a thin rim) and ensure it is well below the ultimate tensile strength of the chosen material, incorporating a significant factor of safety. For high-speed flywheels, detailed stress analysis and robust containment design are critical due to the risk of rotor failure.
    \item \textbf{Check System Dynamics (if needed):} For applications in vehicles or moving platforms, evaluate gyroscopic effects. The gyroscopic moment $T_g$ when the axis is tilted at angular velocity $\omega_p$ is $T_g = I \omega \omega_p$. This can affect vehicle handling and must be considered in the bearing design.
\end{enumerate}

\section{Worked Examples}

\subsection{Example 1: Flywheel for a Single-Cylinder Four-Stroke Engine}
\noindent \textbf{Problem:} For a given engine, analysis of the turning moment diagram yields a maximum fluctuation of energy $\Delta E = 1200$ J. The engine runs at a mean speed of $N = 3000$ rpm. Design a rim-type flywheel with a mean radius of $0.15$ m to limit $C_s$ to $0.02$. Assume a material density of $\rho = 7800$ kg/m$^3$ and a safe working stress of $60$ MPa.

\noindent \textbf{Solution:}
\begin{enumerate}
    \item Mean angular speed: $\omega = (2\pi N)/60 = (2\pi \times 3000)/60 = 314.16$ rad/s.
    \item Required moment of inertia: $I = \Delta E / (\omega^2 C_s) = 1200 / (314.16^2 \times 0.02) \approx 0.608$ kg$\cdot$m$^2$.
    \item For a rim-type flywheel: $I = m r^2$. Required mass: $m = I / r^2 = 0.608 / (0.15)^2 \approx 27.0$ kg.
    \item \textbf{Stress Check:} Tangential velocity $v = \omega r = 314.16 \times 0.15 = 47.12$ m/s. Induced hoop stress $\sigma_t = \rho v^2 = 7800 \times (47.12)^2 \approx 17.3 \times 10^6$ Pa = $17.3$ MPa.
    \item Since the induced stress ($17.3$ MPa) is less than the allowable stress ($60$ MPa), the design is safe. The mass of the flywheel rim is approximately $27.0$ kg.
\end{enumerate}

\subsection{Example 2: Flywheel for a Punching Press}
\noindent \textbf{Problem:} A punching press requires $15,000$ J of energy to punch a hole in $0.1$ seconds. The press is driven by a motor that supplies constant power. The flywheel speed must not drop more than $10\%$ from its nominal speed of $200$ rpm during the punch. Determine the flywheel inertia required. The machine has a $C_s$ of $0.2$.

\noindent \textbf{Solution:}
\begin{enumerate}
    \item The energy for the punch ($\Delta E = 15,000$ J) is supplied by the flywheel's kinetic energy drop.
    \item Mean speed $N = 200$ rpm, so $\omega = (2\pi \times 200)/60 \approx 20.94$ rad/s.
    \item A $10\%$ drop in speed means $\omega_\text{min} = 0.9 \omega = 18.85$ rad/s. Therefore, $C_s = (\omega - \omega_\text{min}) / \omega = 0.1$. (Note: This is the \textit{actual} $C_s$ for the event; the design $C_s$ of the machine is given as $0.2$, which is a larger, safe limit).
    \item Using the stricter event-based requirement ($C_s = 0.1$): $I = \Delta E / (\omega^2 C_s) = 15000 / (20.94^2 \times 0.1) \approx 342$ kg$\cdot$m$^2$.
    \item A flywheel with a moment of inertia of at least $342$ kg$\cdot$m$^2$ is required to limit the speed drop to $10\%$ during the punching operation.
\end{enumerate}

\section{Advanced Concepts and Modern Applications}

\subsection{High-Speed Flywheel Energy Storage Systems (FESS)}
Modern FESS represent the high-tech evolution of the flywheel principle. They are engineered for maximum \textbf{energy density (Wh/kg)} and \textbf{power density (W/kg)}. Key enabling technologies include:
\begin{itemize}
    \item \textbf{Composite Rotors:} Materials like carbon-fiber offer superior strength-to-density ratios, allowing rotational speeds exceeding 50,000 rpm and specific energies over 100 Wh/kg in vacuum.
    \item \textbf{Magnetic Bearings:} Eliminate mechanical contact, drastically reducing friction losses. Systems can achieve a round-trip efficiency (energy out/energy in) of 85-90\%.
    \item \textbf{Vacuum Enclosures:} House the rotor to eliminate windage (air drag) losses, enabling long-term energy storage with minimal decay.
\end{itemize}

\subsection{Applications in Power and Energy Systems}
\begin{itemize}
    \item \textbf{Grid Frequency Regulation and Stability:} FESS can inject or absorb real power in milliseconds, helping to balance supply and demand on the electrical grid, a critical function with increasing renewable penetration.
    \item \textbf{Uninterruptible Power Supply (UPS):} Flywheels provide high-power, short-duration backup power for data centers and critical facilities, often outperforming batteries in cycle life and reliability.
    \item \textbf{Regenerative Braking:} In vehicles and rail systems, kinetic energy during braking is stored in a flywheel and reused for acceleration, improving efficiency.
    \item \textbf{Hybrid Energy Storage Systems:} Flywheels are paired with other technologies to compensate for their weaknesses. For instance, a flywheel provides instant power to cover the 1-2 second start-up lag of a Compressed Air Energy Storage (CAES) system, creating a robust backup power solution.
\end{itemize}

\subsection{Integration with Renewable Energy}
Flywheels are ideal for mitigating the intermittency of wind and solar power. They can smooth the power output from a wind turbine over short periods of gusting or lulls, improving power quality and reducing mechanical stress on the drivetrain.

\subsection{Limitations and Challenges}
Despite advances, challenges remain:
\begin{itemize}
    \item \textbf{Cost:} Advanced materials, magnetic bearings, and vacuum systems are expensive.
    \item \textbf{Self-Discharge:} Even with excellent bearings and vacuum, internal losses cause a gradual slowdown, making FESS unsuitable for long-term (seasonal) storage.
    \item \textbf{Safety and Containment:} Rotor failure at extreme speeds is catastrophic. Robust containment vessels add weight and cost.
    \item \textbf{Gyroscopic Effects:} In vehicular applications, the large angular momentum can create undesirable gyroscopic couples during turning, requiring specialized mounting (gimbals) or management strategies.
\end{itemize}

\section{Conclusion}
The flywheel remains a cornerstone technology in mechanical engineering, embodying the elegant application of rotational dynamics for practical energy management. Its fundamental role in passive governing—the reduction of cyclic speed fluctuations—is governed by the principle of kinetic energy storage and the derived relationship $I = \Delta E / (\omega^2 C_s)$.

From the classic cast-iron wheels smoothing the operation of steam engines to the ultra-high-speed composite rotors in magnetic bearings stabilizing the modern electrical grid, the flywheel has continuously evolved. While its core mathematical formulation is timeless, advances in materials science and mechatronics have unlocked new performance frontiers, transforming it into a key component for high-power, high-cycle energy storage applications.

For the mechanical engineer, a deep understanding of flywheel design principles is essential not only for traditional machine design but also for engaging with cutting-edge challenges in renewable energy integration, transportation efficiency, and grid resilience. The flywheel, in its simplest and most advanced forms, stands as a powerful testament to the enduring utility of inertial energy storage.

\begin{table}[H]
    \centering
    \caption{Nomenclature}
    \label{tab:nomenclature}
    \begin{tabularx}{\textwidth}{|>{\RaggedRight}X|>{\RaggedRight}X|>{\RaggedRight}X|}
        \hline
        \textbf{Symbol} & \textbf{Definition} & \textbf{Typical Units} \\
        \hline
        $C_e$ & Coefficient of fluctuation of energy & - \\
        \hline
        $C_s$ & Coefficient of fluctuation of speed & - \\
        \hline
        $E$, $E_k$ & Kinetic energy & Joule (J) \\
        \hline
        $\Delta E$ & Maximum fluctuation of energy & Joule (J) \\
        \hline
        $I$ & Mass moment of inertia & kg$\cdot$m$^2$ \\
        \hline
        $k$ & Radius of gyration / inertial constant & m / - \\
        \hline
        $m$ & Mass & kilogram (kg) \\
        \hline
        $N$ & Rotational speed & revolutions per minute (rpm) \\
        \hline
        $r$ & Radius & meter (m) \\
        \hline
        $T$ & Torque & Newton-meter (N$\cdot$m) \\
        \hline
        $T_\text{mean}$ & Mean torque & Newton-meter (N$\cdot$m) \\
        \hline
        $v$ & Tangential velocity & meter/second (m/s) \\
        \hline
        $\omega$ & Angular velocity & radian/second (rad/s) \\
        \hline
        $\rho$ & Density & kilogram/meter$^3$ (kg/m$^3$) \\
        \hline
        $\sigma_t$ & Tangential (hoop) stress & Pascal (Pa) \\
        \hline
    \end{tabularx}
\end{table}

\section*{References}
\begin{enumerate}
    \item Wikipedia Contributors. (2026). Flywheel energy storage. Wikipedia. Retrieved January 24, 2026, from \texttt{https://en.wikipedia.org/wiki/Flywheel\_energy\_storage}
    \item Shet, H. (n.d.). Flywheel Torque Fluctuation. Medium. Retrieved January 24, 2026, from \texttt{https://medium.com/@harsh.shet20/fig-torque-time-relation-with-a-flywheel-54fae021e458}
    \item ScienceDirect. (n.d.). Flywheel Energy Storage - an overview. Retrieved January 24, 2026, from \texttt{https://www.sciencedirect.com/topics/engineering/flywheel-energy-storage}
    \item Studysmarter. (2024, August 30). Flywheel Design: Calculation \& Considerations. Retrieved January 24, 2026, from \texttt{https://www.studysmarter.co.uk/explanations/engineering/mechanical-engineering/flywheel-design/}
    \item MITCalc. (n.d.). Design and analysis of the flywheel. Retrieved January 24, 2026, from \texttt{https://www.mitcalc.com/doc/flywheels/help/en/flywheels.htm}
    \item RoyMech. (n.d.). Flywheel Basics. Retrieved January 24, 2026, from \texttt{https://www.roymech.co.uk/Useful\_Tables/Cams\_Springs/Flywheels.html}
    \item Smith, B. (2021, February 2). Applications of Compressed Air and Flywheel Combinations in Energy Storage. AZoM. Retrieved January 24, 2026, from \texttt{https://www.azom.com/article.aspx?ArticleID=18881}
    \item The Engineering ToolBox. (n.d.). Flywheels - Kinetic Energy. Retrieved January 24, 2026, from \texttt{https://www.engineeringtoolbox.com/flywheel-energy-d\_945.html}
\end{enumerate}

\end{document}