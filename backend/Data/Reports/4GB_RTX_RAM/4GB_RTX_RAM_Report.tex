\documentclass[12pt, a4paper]{article}
\usepackage[utf8]{inputenc}
\usepackage[T1]{fontenc}
\usepackage{geometry}
\geometry{margin=1in}
\usepackage{setspace}
\onehalfspacing
\usepackage{amsmath, amssymb}
\usepackage{graphicx}
\usepackage{tabularx}
\usepackage{booktabs}
\usepackage{microtype}
\usepackage{ragged2e}
\usepackage[colorlinks=true, allcolors=blue]{hyperref}
\usepackage[parfill]{parskip}
\usepackage{float}

\title{A Comprehensive Technical Analysis of 4GB VRAM Configurations in NVIDIA RTX Graphics Solutions}
\author{Subhash Chandra \\ Govind Ballabh Pant Institute of Engineering \& Technology (GBPIET), Ghurdauri \\ MTech (Thermal Engineering)}
\date{January 27, 2026}

\begin{document}

\maketitle

\begin{abstract}
This report presents a critical examination of NVIDIA RTX graphics processing units (GPUs) configured with 4 gigabytes (GB) of Video Random Access Memory (VRAM). The RTX brand, synonymous with real-time ray tracing and artificial intelligence (AI)-accelerated rendering, represents a paradigm shift in visual computing. However, this analysis posits that a 4GB VRAM buffer creates a fundamental and often crippling architectural contradiction, severely limiting the practical utility of the very advanced features these GPUs are designed to execute. The core thesis is that while 4GB RTX GPUs exist as legitimate products within NVIDIA's professional and consumer lineups, they are niche solutions suited only for specific, lightweight workflows. Their performance is critically bottlenecked in modern gaming, professional content creation, and scientific computing applications, where VRAM requirements routinely exceed 6-8GB. The primary bottleneck is not merely capacity but the intrinsically linked \textbf{memory bandwidth}, dictated by a narrow bus interface (typically 64-bit or 128-bit), which fails to feed the parallel architecture of the GPU core efficiently.
\end{abstract}

\tableofcontents
\newpage

\section{Introduction: The RTX Paradigm and the VRAM Contradiction}
The introduction of the NVIDIA RTX platform marked a revolutionary shift from traditional rasterization-based graphics to a hybrid model incorporating dedicated hardware for ray tracing (RT Cores) and AI-based processing (Tensor Cores). This architecture enables real-time simulation of physically accurate lighting, reflections, and shadows, along with performance-enhancing technologies like Deep Learning Super Sampling (DLSS).

The foundational promise of RTX is an enhanced, immersive visual experience. However, this promise is contingent upon a graphics subsystem capable of managing exponentially more complex data. Ray tracing requires the construction and traversal of Bounding Volume Hierarchy (BVH) structures—spatial acceleration data that can consume several gigabytes of VRAM on its own. Similarly, AI workloads, including DLSS, require space for neural network weights and intermediate tensors.

Concurrently, the general trend in software development has been toward higher-fidelity assets. Game textures for 4K resolution can demand 8GB or more; professional engineering models and scientific datasets regularly exceed 16GB. In this landscape, a 4GB VRAM pool is extraordinarily limiting.

This report will analyze this conflict through the following lenses: the technical specifications of existing 4GB RTX solutions, the mathematical models governing memory performance, empirical data from gaming and professional applications, and the thermodynamic implications of the typically low-power design of these cards. The primary subject of analysis will be the professional \textbf{NVIDIA RTX A400} and the consumer \textbf{GeForce RTX 3050 4GB}, with references to the embedded \textbf{RTX A500/A1000} where applicable.

\section{Technical Deep Dive: Architecture, Specifications, and the Bottleneck}
\subsection{Product Spectrum and Core Specifications}
NVIDIA deploys 4GB VRAM configurations across several market segments. The specifications reveal a consistent pattern of constrained memory subsystems.

\begin{table}[H]
\centering
\caption{Comparative Specifications of Select 4GB RTX GPUs and Higher-VRAM Counterparts}
\label{tab:specs}
\begin{tabularx}{\textwidth}{l l X c c c c c}
\toprule
\textbf{Model} & \textbf{Market Segment} & \textbf{Architecture} & \textbf{CUDA Cores} & \textbf{VRAM} & \textbf{Mem. Interface} & \textbf{Mem. Bandwidth} & \textbf{TGP (W)} \\
\midrule
RTX A400 & Professional Workstation & Ampere & 768 & 4GB GDDR6 & 64-bit & 96 GB/s & 50 \\
RTX A1000 & Embedded / Professional & Ampere & 2,048 & 4GB GDDR6 & 128-bit & 224 GB/s & 60 \\
RTX 3050 4GB & Consumer Laptop & Ampere/Ada & 1,792-2,048 & 4GB GDDR6 & 64/128-bit & ~96-224 GB/s & 35-80 \\
RTX A2000 8GB & Professional Workstation & Ampere & 2,560 & 8GB GDDR6 & 224-bit & 224 GB/s & 60 \\
RTX 4060 8GB & Consumer Desktop & Ada Lovelace & 3,072 & 8GB GDDR6 & 128-bit & 272 GB/s & 115 \\
\bottomrule
\end{tabularx}
\end{table}

\noindent \textbf{Key Observations from Table \ref{tab:specs}:}
\begin{itemize}
    \item \textbf{Narrow Memory Interface:} The RTX A400 and certain RTX 3050 models utilize a \textbf{64-bit memory bus}. This is a primary bottleneck, as bandwidth is directly proportional to bus width.
    \item \textbf{Severely Limited Bandwidth:} The RTX A400's 96 GB/s bandwidth is a fraction of mainstream GPUs (e.g., RTX 4060 at 272 GB/s). This restricts the rate at which textures, geometry data, and BVH structures can be fed to the cores.
    \item \textbf{Core Count Parity Misleading:} The RTX A1000 has a respectable 2,048 CUDA cores, identical to the 8GB RTX A2000. However, its 4GB/128-bit configuration means these powerful cores will frequently sit idle, waiting for data, a condition known as "starving the pipeline."
\end{itemize}

\subsection{The Mathematical Foundation of Memory Performance}
The performance of a GPU's memory subsystem is quantifiable through two key formulas.

\subsubsection{Theoretical Peak Memory Bandwidth}
This is the maximum achievable data transfer rate between the GPU and its VRAM. It is calculated as: \( \text{Bandwidth} = (\text{Memory Clock Frequency}) \times (\text{Memory Bus Width}) \times (\text{Memory Type Factor}) \). For modern GDDR6/GDDR6X memory, the factor accounts for data rate per pin (e.g., GT/s - Giga Transfers per second). A more precise formulation is:
$$ BW = f_{\text{mem}} \times I_{\text{bus}} \times \left( \frac{DR}{8} \right) $$
Where: \( BW \) = Bandwidth in GB/s, \( f_{\text{mem}} \) = Memory Clock frequency in MHz, \( I_{\text{bus}} \) = Memory Bus Interface width in bits, \( DR \) = Data Rate per pin (e.g., 16 Gb/s for standard GDDR6). The division by 8 converts gigabits to gigabytes.

Applying the formula to the RTX A400: With a stated bandwidth of 96 GB/s on a 64-bit bus, we can infer its effective memory data rate. Conversely, a card like the RTX A2000 8GB achieves 224 GB/s with a 224-bit bus and similar GDDR6 technology, showcasing how a wider bus directly multiplies available bandwidth.

\subsubsection{VRAM Capacity Requirement Estimation (Simplified Model)}
While dependent on the application, a basic model for a gaming workload can be conceptualized:
$$ V_{\text{req}} \approx (R_T \times S_T) + (F_B \times R_X \times R_Y \times B_D) + V_{\text{BVH}} + V_{\text{OS}} $$
Where: \( V_{\text{req}} \) = Estimated total VRAM required (in bytes), \( R_T \) = Number of resident textures, \( S_T \) = Average size per texture (e.g., 4MB for a 2K compressed texture), \( F_B \) = Number of framebuffers (color, depth, G-buffer), \( R_X, R_Y \) = Display resolution (pixels), \( B_D \) = Bit depth per pixel per buffer (e.g., 32 bits = 4 bytes), \( V_{\text{BVH}} \) = Memory for Ray Tracing BVH structures (can be 1-3 GB), \( V_{\text{OS}} \) = Operating System and driver overhead (typically 0.5-1 GB).

For a modern game at 1920x1080 with high-quality textures and ray tracing enabled, \( V_{\text{req}} \) can easily exceed 6GB, immediately disqualifying a 4GB GPU from a stutter-free experience. This model aligns with empirical data from games where 6GB was found to be insufficient, causing severe stutters, and 8GB was recommended for stable performance.

\subsection{Advanced Memory Technologies: GDDR6 vs. GDDR6X}
The type of memory used significantly impacts the bandwidth equation. Most 4GB RTX cards utilize standard GDDR6. The more advanced GDDR6X, used in higher-end RTX 30/40 series cards, employs PAM4 (Pulse Amplitude Modulation 4-level) signaling.
\begin{itemize}
    \item \textbf{GDDR6 (NRZ/PAM2):} Transmits 1 bit per clock cycle (0 or 1). To increase bandwidth, the clock speed must increase, raising power consumption and signal integrity challenges.
    \item \textbf{GDDR6X (PAM4):} Transmits 2 bits per clock cycle (00, 01, 10, 11), effectively doubling the data rate at the same clock frequency. This is represented as: $$ \text{DataRate}_{\text{PAM4}} \approx 2 \times \text{DataRate}_{\text{NRZ}} \quad \text{(at similar clock speeds)} $$
\end{itemize}
The absence of GDDR6X in 4GB configurations is both a cost-saving measure and a technical acknowledgment that the narrow bus width is the dominant bottleneck; enhancing the memory type would not alleviate the fundamental capacity and interface constraints.

\section{Performance Analysis: Synthetic Benchmarks and Real-World Limitations}
\subsection{The Gaming Performance Quagmire}
Gaming is the most visible arena where the 4GB limitation manifests, often catastrophically.
\begin{itemize}
    \item \textbf{1080p Esports \& Legacy Titles:} In games like CS:GO or Valorant, a 4GB RTX GPU can deliver high frame rates. The graphical demands are low, and the RTX branding is largely irrelevant here.
    \item \textbf{Modern AAA Titles (2020-Present):} This is where the configuration fails. Games such as Hogwarts Legacy or Cyberpunk 2077 routinely allocate 7-9GB of VRAM at 1080p High settings. When VRAM is exhausted, the system must page assets to and from slower system RAM over the PCIe bus. This causes massive spikes in frame render time, perceived as severe stuttering or freezing, even if the average FPS seems acceptable.
    \item \textbf{The Ray Tracing Paradox:} Enabling ray tracing in any demanding title instantly increases \( V_{\text{BVH}} \) and often uses higher-resolution assets for ray-traced reflections. This makes the feature practically unusable on a 4GB card, as it will either crash, stutter uncontrollably, or force such drastic reductions in other quality settings that the visual benefit is nullified.
\end{itemize}

\subsection{Professional \& Scientific Computing Workloads}
For the professional user, VRAM is a critical workspace. The RTX A400 is marketed for CAD, digital content creation (DCC), and light AI workflows.
\begin{itemize}
    \item \textbf{Computer-Aided Design (CAD):} Simple parts and 2D drafting are feasible. However, navigating a complex assembly (e.g., an entire engine or aircraft subsystem) requires the model data to reside in VRAM for smooth manipulation. With 4GB, users will encounter lag, slow viewport rotation, and frequent loading pauses as data is swapped.
    \item \textbf{Data Science \& AI:} VRAM capacity is the single most important factor for model training. It dictates the maximum batch size and model complexity that can be handled. Even moderate deep learning models require 12-24GB of VRAM. A 4GB card is relegated to running small-scale inference or educational demonstrations on tiny datasets. Its Tensor Cores are fundamentally underutilized.
    \item \textbf{3D Rendering \& Simulation:} Rendering a complex scene in Blender (Cycles) or running a computational fluid dynamics (CFD) mesh requires storing geometry, textures, and solver data in VRAM. The 4GB limit forces drastic reductions in scene complexity, resolution, or simulation fidelity, undermining the purpose of GPU acceleration.
\end{itemize}

\subsection{Case Study: The "Minimum Spec" Deception}
The game \emph{Dying Light: The Beast} officially listed a minimum requirement of 6GB VRAM. However, performance testing revealed that 6GB was insufficient, causing "random, very heavy stutters" that could last 1-2 seconds on an RTX 4050 laptop GPU. The tester concluded that 8GB should be considered the true minimum for a playable experience. This case exemplifies the growing gap between official minimum specifications and real-world, stutter-free performance requirements—a gap that utterly consumes a 4GB GPU.

\section{Thermal and Power Design Considerations}
A defining characteristic of many 4GB RTX solutions is low Total Graphics Power (TGP). The RTX A400 has a 50W TGP, and the mobile RTX 3050 can be as low as 35W. This is both a constraint and a design advantage.
\begin{itemize}
    \item \textbf{Thermal Design Power (TDP) as a Limiter:} The low TGP is a direct enabler of the single-slot, low-profile form factor seen in cards like the RTX A400. It allows integration into slim workstations, embedded systems, and silent PCs. However, this power budget also dictates the performance ceiling. The GPU cannot sustain high core or memory clocks without exceeding its thermal and electrical limits.
    \item \textbf{Cooling Solution Implications:} With a 50W heat load, a simple aluminum heatsink and a small, low-RPM fan are sufficient. This minimizes acoustic noise—a critical factor in office environments. From a thermal engineering perspective, the design priority is reliability and longevity over peak heat dissipation. The compact design also simplifies integration into custom or industrial chassis where space is limited.
    \item \textbf{Efficiency vs. Performance:} Newer architectures like Ada Lovelace (used in the RTX 3050 A) offer better performance-per-watt than older Ampere designs. This means that within the same strict 35-50W power envelope, an Ada-based 4GB GPU could deliver slightly better performance than its Ampere predecessor, but the VRAM bottleneck remains the dominant limiting factor.
\end{itemize}

\section{Market Positioning, Use Cases, and Strategic Rationale}
Given its severe limitations, why does the 4GB RTX segment exist? Its positioning is highly specialized.
\begin{itemize}
    \item \textbf{Legacy System Upgrades \& OEM SFF PCs:} For businesses with older, small-form-factor desktops that lack a powerful power supply, a low-profile, sub-75W card like the RTX A400 provides a modern display output (4x DisplayPort 1.4) and GPU acceleration for video decode/encode and basic 3D viewing without requiring a hardware overhaul.
    \item \textbf{Multi-Display Business Workstations:} In financial trading, control rooms, or basic design offices, the primary need is driving multiple 4K/5K displays reliably. The 4GB VRAM is adequate for the desktop composition buffer, and the RTX A400's four native display outputs make it a cost-effective solution.
    \item \textbf{Embedded and Edge Applications:} The RTX A1000/A500 are designed for embedded markets like medical imaging, aerospace, and digital signage. Here, the requirements are long-term reliability, specific I/O, and low power consumption within a compact module. The 3D and AI capabilities, though limited by VRAM, provide a standardized platform for running specialized visualization algorithms.
    \item \textbf{NVIDIA's Product Strategy:} From a silicon economics perspective, a GPU like the mobile RTX 3050 A is created using "down-binned" dies where some cores or memory controllers are defective. By disabling these non-functional parts, NVIDIA can salvage silicon that would otherwise be discarded, creating a viable product for the low-power, thin-and-light laptop market. It represents efficient use of manufacturing yield rather than a performance-targeted design.
\end{itemize}

\begin{table}[H]
\centering
\caption{Appropriate vs. Inappropriate Use Cases for 4GB RTX GPUs}
\label{tab:usecases}
\begin{tabularx}{\textwidth}{X X}
\toprule
\textbf{Appropriate Use Cases} & \textbf{Inappropriate Use Cases} \\
\midrule
Driving 2-4 high-resolution displays for desktop applications. & Modern AAA gaming at 1080p Medium/High settings. \\
Basic CAD viewing and editing of small-to-medium part files. & Ray-traced gaming or rendering. \\
Hardware-accelerated video decode/encode (H.264, H.265, AV1) for streaming. & Training deep learning/AI models of meaningful size. \\
Legacy system upgrade where power and space are severely constrained. & Professional 3D content creation (modeling, sculpting, scene assembly). \\
Embedded vision systems in healthcare, retail, or industrial settings. & High-resolution (4K+) video editing and color grading. \\
\bottomrule
\end{tabularx}
\end{table}

\section{Conclusion and Future Outlook}
The 4GB RTX GPU is a study in technological compromise. It packages an advanced, parallel computing architecture—capable of ray tracing and AI acceleration—within a memory subsystem that is fundamentally inadequate to unleash its potential for mainstream demanding tasks. The \textbf{architectural dissonance} between the capabilities of the SM (Streaming Multiprocessor), RT Core, and Tensor Core arrays and the constraints of the 64/128-bit memory bus and 4GB capacity is the central finding of this report.

For the \textbf{MTech (Thermal Engineering)} perspective, these GPUs are exemplary models of constrained thermal design, achieving functionality within a strict 35-60W power envelope, enabling passive or near-silent active cooling solutions suitable for sensitive environments.

However, the trendline is clear. Software demands for VRAM are increasing monotonically due to higher-resolution assets, larger datasets, and more complex simulation techniques. Technologies like Unreal Engine 5's Nanite and Lumen further accelerate this demand. Consequently, the useful lifespan of a 4GB GPU in any performance-sensitive context is extremely short.

\noindent \textbf{Final Recommendation:} Users should interpret the "4GB RTX" configuration as a \textbf{specialized tool for specific scenarios}, not a general-purpose graphics solution. For students, professionals, or enthusiasts whose workflows involve modern games, engineering simulation, content creation, or data science, an investment in a GPU with at least \textbf{8GB of VRAM}—and a wider memory bus—is not merely advisable; it is essential to avoid immediate performance bottlenecks and ensure a viable system for the coming 2-3 years. The 4GB RTX serves its niche but stands in stark contrast to the expansive promise of the RTX brand itself.

\section*{References}
\begin{enumerate}
    \item NVIDIA. (n.d.). \textit{NVIDIA RTX Embedded GPU Solutions}. NVIDIA Corporation.
    \item Leadtek. (n.d.). \textit{NVIDIA RTX A400 | NVIDIA Professional Graphics}. Leadtek Research Inc.
    \item amrits. (2023, March 29). \textit{VRAM Allocation Issues - Linux}. NVIDIA Developer Forums.
    \item Wevolver. (2025, October 6). \textit{GDDR6 vs GDDR6X: A Comprehensive Technical Comparison for Digital Design \& Hardware Engineers}. Wevolver.
    \item Oladapo, O. (2025, June 30). \textit{How Much GPU Memory Do You Need in a Data Science Workstation?}. HP Tech Takes.
    \item Evanson, N. (2025). \textit{Dying Light: The Beast PC performance analysis: Decent frame rates all round, nice graphics, and stutters only on low VRAM GPUs}. PC Gamer.
    \item Butts, J. (2025). \textit{Nvidia RTX 3050 A Laptop GPU specs revealed and it's as weak as expected — comes with just 1,768 CUDA cores and 4GB VRAM on a 64-bit bus}. Tom's Hardware.
\end{enumerate}

\end{document}